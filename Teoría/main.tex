\documentclass[osajnl,showpacs,superscriptaddress,10pt]{revtex4-2}

%-------------------------------------------------
% ------------------- PAQUETES -------------------
%-------------------------------------------------
\usepackage{bm,bbm,lipsum}
\usepackage[spanish,es-tabla]{babel}
\usepackage[utf8]{inputenc}
\usepackage[T1]{fontenc}
\usepackage{graphicx,animate,xcolor,subfig,gensymb,physics,amsmath,amsfonts,amssymb,mathrsfs,pbsi,calligra}
\usepackage[pdftex]{hyperref}
\usepackage{multirow,booktabs,float,wrapfig2,boxedminipage,tensor}
\usepackage{titlesec,indentfirst,calc,changepage,extpfeil}
\usepackage[skins,breakable,most]{tcolorbox}
\tcbuselibrary{breakable}
\usepackage[dvipsnames]{xcolor}
\usepackage{floatflt,subcaption,textcomp}
\AtBeginDocument{\selectlanguage{spanish}}
\decimalpoint

%-------------------------------------------------
% --------- CONFIGURACIONES Y MACROS -------------
%-------------------------------------------------
\numberwithin{equation}{section} % (1.1), (1.2), etc.
\setlength{\parindent}{12pt}
\setlength{\parskip}{4pt}

% --- Definiciones de colores y cajas ---
\tcbset{
  bluebox/.style={
    colback=blue!5!white,
    colframe=blue!60!black,
    boxrule=0.5pt,
    sharp corners,
    width=\linewidth,
    breakable,
    enhanced jigsaw,
    before upper=\color{black},
  }
}
\lstset{
    basicstyle=\ttfamily\footnotesize,
    keywordstyle=\color{blue},
    commentstyle=\color{gray},
    stringstyle=\color{red},
    showstringspaces=false,
    breaklines=true
}

\newcommand{\comentario}[1]{\textcolor{gray}{\textit{#1}}}

%-------------------------------------------------
% ------------------- DOCUMENTO ------------------
%-------------------------------------------------
\begin{document}

\title{Oscilador Armónico Cuántico y Estados Coherentes}
\maketitle

% \begin{abstract}
%     Este trabajo presenta un análisis detallado del oscilador armónico cuántico y su extensión hacia los estados coherentes, abordando la formulación clásica y cuántica, los operadores de creación y aniquilación, la representación en el espacio de Fock y el papel fundamental de estos conceptos en la óptica cuántica moderna. Se incluyen desarrollos matemáticos explicados paso a paso, con apoyo en obras clásicas y contemporáneas de la disciplina.
% \end{abstract}

\tableofcontents
\newpage

%-------------------------------------------------
% ------------------ CAPÍTULOS -------------------
%-------------------------------------------------
\section{Introducción}
El oscilador armónico es uno de los sistemas más fundamentales tanto en física clásica como en mecánica cuántica. Su análisis permite comprender el comportamiento de una amplia variedad de sistemas físicos: desde la vibración de un átomo en una molécula hasta el modo de un campo electromagnético confinado en una cavidad óptica.
\comentario{En esta sección se presenta el contexto histórico y la motivación del estudio.}
\begin{tcolorbox}[bluebox]
\begin{equation}
  \begin{aligned}
    H &= \frac{p^2}{2m} + \frac{1}{2}m\omega^2x^2
  \end{aligned}
\end{equation}
\end{tcolorbox}
\comentario{Esta ecuación representa el Hamiltoniano clásico del oscilador armónico. El primer término corresponde a la energía cinética y el segundo a la energía potencial.}
Para pasar al régimen cuántico, se promueven las variables $x$ y $p$ a operadores, de modo que:
\begin{tcolorbox}[bluebox]
\begin{equation}
  \begin{aligned}
    \hat{x} &= x, \qquad \hat{p} = -i\hbar \frac{\partial}{\partial x}
  \end{aligned}
\end{equation}
\end{tcolorbox}
\comentario{De esta manera, el Hamiltoniano cuántico toma la forma}
\begin{tcolorbox}[bluebox]
\begin{equation}
  \begin{aligned}
    \hat{H} = \frac{\hat{p}^2}{2m} + \frac{1}{2}m\omega^2\hat{x}^2
  \end{aligned}
\end{equation}
\end{tcolorbox}

\section{Oscilador Armónico Clásico}
El oscilador armónico clásico describe el movimiento de una partícula sujeta a una fuerza restauradora proporcional a su desplazamiento. Matemáticamente, esta relación se expresa mediante la \textit{segunda ley de Newton}:
\begin{tcolorbox}[bluebox]
\begin{equation}
  \begin{aligned}
    F &= -k\,x
  \end{aligned}
\end{equation}
\end{tcolorbox}
\comentario{La constante $k$ caracteriza la rigidez del sistema. Cuanto mayor sea $k$, más intensa será la fuerza de restitución.}
Aplicando la relación de Newton $F = m\,a$, obtenemos la ecuación diferencial del movimiento:
\begin{tcolorbox}[bluebox]
\begin{equation}
  \begin{aligned}
    m\frac{d^2x}{dt^2} + kx = 0
  \end{aligned}
\end{equation}
\end{tcolorbox}
Esta ecuación tiene como solución general:
\begin{tcolorbox}[bluebox]
\begin{equation}
  \begin{aligned}
    x(t) = A\cos(\omega t + \phi)
  \end{aligned}
\end{equation}
\end{tcolorbox}
donde $\omega = \sqrt{k/m}$ es la \textit{frecuencia angular}, $A$ la \textit{amplitud} y $\phi$ la \textit{fase inicial}.
\comentario{El movimiento es periódico, y la energía total del sistema se mantiene constante.}
\\La energía total se expresa mediante el Hamiltoniano clásico:
\begin{tcolorbox}[bluebox]
\begin{equation}
  \begin{aligned}
    H &= \frac{p^2}{2m} + \frac{1}{2}m\omega^2x^2
  \end{aligned}
\end{equation}
\end{tcolorbox}
donde $p = m\,\dot{x}$ es el momento lineal. Este Hamiltoniano servirá como punto de partida para la formulación cuántica.

\section{Oscilador Armónico Cuántico}
Al cuantizar el oscilador armónico, se sustituyen las variables clásicas $x$ y $p$ por operadores $\hat{x}$ y $\hat{p}$ que cumplen la relación de conmutación fundamental:
\begin{tcolorbox}[bluebox]
\begin{equation}
  \begin{aligned}
    [\hat{x}, \hat{p}] = i\hbar
  \end{aligned}
\end{equation}
\end{tcolorbox}
\comentario{Esta relación refleja el principio de incertidumbre de Heisenberg.}
El Hamiltoniano cuántico se obtiene directamente del clásico:
\begin{tcolorbox}[bluebox]
\begin{equation}
  \begin{aligned}
    \hat{H} = \frac{\hat{p}^2}{2m} + \frac{1}{2}m\omega^2\hat{x}^2
  \end{aligned}
\end{equation}
\end{tcolorbox}
Para simplificar el análisis, se introducen los operadores de creación y aniquilación:
\begin{tcolorbox}[bluebox]
\begin{equation}
  \begin{aligned}
    \hat{a} &= \sqrt{\frac{m\omega}{2\hbar}}\left(\hat{x} + \frac{i}{m\omega}\hat{p}\right), \\
    \hat{a}^\dagger &= \sqrt{\frac{m\omega}{2\hbar}}\left(\hat{x} - \frac{i}{m\omega}\hat{p}\right)
  \end{aligned}
\end{equation}
\end{tcolorbox}
\comentario{Estos operadores no tienen análogo clásico: $\hat{a}$ destruye un cuanto de energía, mientras que $\hat{a}^\dagger$ crea uno.}
Reescribiendo el Hamiltoniano en términos de ellos:
\begin{tcolorbox}[bluebox]
\begin{equation}
  \begin{aligned}
    \hat{H} &= \hbar\omega\left(\hat{a}^\dagger\hat{a} + \frac{1}{2}\right)
  \end{aligned}
\end{equation}
\end{tcolorbox}
\comentario{La expresión anterior revela que la energía del oscilador cuántico está cuantizada.}
Los autovalores de energía se obtienen aplicando $\hat{H}$ sobre los estados de número $|n\rangle$:
\begin{tcolorbox}[bluebox]
\begin{equation}
  \begin{aligned}
    \hat{H}|n\rangle &= E_n|n\rangle, \quad E_n = \hbar\omega\left(n+\frac{1}{2}\right)
  \end{aligned}
\end{equation}
\end{tcolorbox}
\comentario{El valor $\frac{1}{2}\hbar\omega$ representa la energía del punto cero, consecuencia directa de la imposibilidad de anular simultáneamente $x$ y $p$.}

\section{Estados Coherentes}
Los \textit{estados coherentes} son soluciones especiales del oscilador armónico cuántico que reproducen, en el límite cuántico, el comportamiento de una oscilación clásica.  
Fueron introducidos formalmente por Roy J. Glauber en la década de 1960 como la base teórica de la \textit{óptica cuántica} moderna.
\comentario{Un estado coherente puede considerarse el análogo cuántico de una onda clásica monocromática.}

\subsection{Definición del estado coherente}
El estado coherente $|\alpha\rangle$ se define como un \textit{autovector del operador de aniquilación} $\hat{a}$:
\begin{tcolorbox}[bluebox]
\begin{equation}
  \begin{aligned}
    \hat{a}|\alpha\rangle = \alpha |\alpha\rangle
  \end{aligned}
\end{equation}
\end{tcolorbox}
donde $\alpha$ es un número complejo que codifica tanto la amplitud como la fase del estado.
\comentario{A diferencia de los estados de número $|n\rangle$, los estados coherentes no tienen un número definido de cuantos, sino una distribución de Poisson en energía.}
La expansión de $|\alpha\rangle$ en la base de Fock se obtiene aplicando el operador de desplazamiento:
\begin{tcolorbox}[bluebox]
\begin{equation}
  \begin{aligned}
    |\alpha\rangle = e^{-|\alpha|^2/2}\sum_{n=0}^{\infty}\frac{\alpha^n}{\sqrt{n!}}|n\rangle
  \end{aligned}
\end{equation}
\end{tcolorbox}
\comentario{El coeficiente exponencial garantiza la normalización $\langle \alpha|\alpha\rangle = 1$.}

\subsection{Propiedades principales}
Los estados coherentes poseen propiedades análogas a las de una onda clásica:
\begin{enumerate}
  \item Minimiza el principio de incertidumbre de Heisenberg:
  \begin{tcolorbox}[bluebox]
  \begin{equation}
    \begin{aligned}
      \Delta x\,\Delta p = \frac{\hbar}{2}
    \end{aligned}
  \end{equation}
  \end{tcolorbox}
  \comentario{Esto significa que las incertidumbres en posición y momento están igualmente repartidas.}
  \item La media de los operadores $\hat{x}$ y $\hat{p}$ evoluciona como en el caso clásico:
  \begin{tcolorbox}[bluebox]
  \begin{equation}
    \begin{aligned}
      \langle \hat{x}(t) \rangle &= \sqrt{\frac{2\hbar}{m\omega}}\Re[\alpha e^{-i\omega t}], \\
      \langle \hat{p}(t) \rangle &= \sqrt{2\hbar m\omega}\Im[\alpha e^{-i\omega t}]
    \end{aligned}
  \end{equation}
  \end{tcolorbox}
  \comentario{Ambas expectativas oscilan con frecuencia $\omega$, reproduciendo el movimiento clásico.}
\end{enumerate}

\subsection{Operador de desplazamiento}
Otra forma equivalente de construir $|\alpha\rangle$ es mediante el operador de desplazamiento:
\begin{tcolorbox}[bluebox]
\begin{equation}
  \begin{aligned}
    \hat{D}(\alpha) &= e^{\alpha\hat{a}^\dagger - \alpha^*\hat{a}}, \\
    |\alpha\rangle &= \hat{D}(\alpha)|0\rangle
  \end{aligned}
\end{equation}
\end{tcolorbox}
\comentario{El operador $\hat{D}(\alpha)$ desplaza el estado de vacío en el espacio de fase, generando el estado coherente correspondiente.}

\subsection{Interpretación en el espacio de fase}
En la representación de Wigner, el estado coherente aparece como una distribución gaussiana centrada en el punto $(x_0, p_0)$ del espacio de fase:
\begin{tcolorbox}[bluebox]
\begin{equation}
  \begin{aligned}
    W_\alpha(x,p) = \frac{1}{\pi\hbar}\exp\left[-\frac{(x-x_0)^2}{\sigma_x^2} - \frac{(p-p_0)^2}{\sigma_p^2}\right]
  \end{aligned}
\end{equation}
\end{tcolorbox}
\comentario{Esta forma resalta la dualidad del estado coherente: una función cuántica que conserva la localización mínima permitida por el principio de incertidumbre.}

\subsection{Energía media y dispersión}
El valor medio de la energía en el estado $|\alpha\rangle$ es:
\begin{tcolorbox}[bluebox]
\begin{equation}
  \begin{aligned}
    \langle \hat{H} \rangle_\alpha &= \hbar\omega\left(|\alpha|^2 + \frac{1}{2}\right)
  \end{aligned}
\end{equation}
\end{tcolorbox}
\comentario{La energía media se comporta como en un oscilador clásico, con un término adicional correspondiente a la energía del punto cero.}

\section{Operadores de Cuadratura}
En óptica cuántica, los \textit{operadores de cuadratura} proporcionan una forma conveniente de describir los campos electromagnéticos en términos análogos a las variables de posición y momento del oscilador armónico cuántico.
\comentario{Las cuadraturas permiten expresar las observables medibles de un campo cuántico en un marco geométrico y probabilístico.}

\subsection{Definición de las cuadraturas}
Dado un modo del campo descrito por los operadores de creación y aniquilación $\hat{a}$ y $\hat{a}^{\dagger}$, se definen las cuadraturas canónicas:
\begin{tcolorbox}[bluebox]
\begin{equation}
  \begin{aligned}
    \hat{X} &= \frac{1}{\sqrt{2}}\left(\hat{a} + \hat{a}^\dagger\right), \\
    \hat{P} &= \frac{1}{i\sqrt{2}}\left(\hat{a} - \hat{a}^\dagger\right)
  \end{aligned}
\end{equation}
\end{tcolorbox}
\comentario{Los operadores $\hat{X}$ y $\hat{P}$ cumplen el mismo álgebra de conmutación que las variables de posición y momento:}
\begin{tcolorbox}[bluebox]
\begin{equation}
  \begin{aligned}
    [\hat{X}, \hat{P}] = i
  \end{aligned}
\end{equation}
\end{tcolorbox}
\comentario{En unidades naturales con $\hbar = 1$, la relación anterior coincide con la de Heisenberg.}

\subsection{Expresiones en términos de operadores físicos}
El Hamiltoniano del oscilador puede reescribirse en función de las cuadraturas:
\begin{tcolorbox}[bluebox]
\begin{equation}
  \begin{aligned}
    \hat{H} &= \frac{\hbar\omega}{2}\left(\hat{X}^2 + \hat{P}^2\right)
  \end{aligned}
\end{equation}
\end{tcolorbox}
\comentario{Esta forma revela la equivalencia entre la energía del campo y la suma cuadrática de las amplitudes conjugadas.}

\subsection{Cuadratura generalizada}
Se define la \textit{cuadratura generalizada} para un ángulo de fase $\theta$ como:
\begin{tcolorbox}[bluebox]
\begin{equation}
  \begin{aligned}
    \hat{X}_\theta = \frac{1}{\sqrt{2}}\left(\hat{a}e^{-i\theta} + \hat{a}^\dagger e^{i\theta}\right)
  \end{aligned}
\end{equation}
\end{tcolorbox}
\comentario{El parámetro $\theta$ permite seleccionar la componente del campo que se mide en un experimento de homodinia o heterodinia.}
\\Las cuadraturas $\hat{X}_\theta$ y $\hat{X}_{\theta+\pi/2}$ forman un par conjugado que obedece la relación de conmutación:
\begin{tcolorbox}[bluebox]
\begin{equation}
  \begin{aligned}
    [\hat{X}_\theta, \hat{X}_{\theta+\pi/2}] = i
  \end{aligned}
\end{equation}
\end{tcolorbox}

\subsection{Propiedades estadísticas}
La dispersión de una cuadratura en un estado cualquiera $|\psi\rangle$ se define como:
\begin{tcolorbox}[bluebox]
\begin{equation}
  \begin{aligned}
    (\Delta X_\theta)^2 = \langle \psi | \hat{X}_\theta^2 | \psi \rangle - \langle \psi | \hat{X}_\theta | \psi \rangle^2
  \end{aligned}
\end{equation}
\end{tcolorbox}
En el caso de un estado coherente $|\alpha\rangle$, se cumple:
\begin{tcolorbox}[bluebox]
\begin{equation}
  \begin{aligned}
    (\Delta X_\theta)^2 = \frac{1}{2}
  \end{aligned}
\end{equation}
\end{tcolorbox}
\comentario{Esto indica que las cuadraturas de un estado coherente poseen incertidumbre mínima y simétrica, reproduciendo el límite clásico.}

\subsection{Interpretación física}
Los operadores $\hat{X}$ y $\hat{P}$ representan las componentes de fase del campo eléctrico cuántico. En una medición experimental, cada cuadratura puede ser accedida mediante técnicas de detección homodina.
\comentario{En capítulos posteriores se mostrará cómo la compresión cuántica modifica la dispersión de las cuadraturas, dando origen a los estados \textit{squeezed}.}

\section{Espacio de Fase Cuántico}
El \textit{espacio de fase cuántico} constituye una representación intermedia entre la descripción de operadores y la descripción por funciones de onda.  
En él, un estado cuántico se expresa como una distribución sobre las variables continuas de posición y momento (o sus equivalentes en cuadraturas).
\comentario{El espacio de fase permite visualizar el estado cuántico de manera análoga a las distribuciones clásicas, pero respetando las restricciones impuestas por el principio de incertidumbre.}

\subsection{Motivación y concepto general}
En mecánica clásica, el estado de una partícula está completamente determinado por un punto $(x,p)$ en el espacio de fase.  
En mecánica cuántica, la imposibilidad de definir simultáneamente $x$ y $p$ con precisión obliga a reemplazar el punto por una \textit{función cuasi-probabilidad}.

\subsection{Función de Wigner}
La \textit{función de Wigner}, introducida en 1932, es la representación más común en el espacio de fase cuántico.  
Se define para un estado descrito por la función de onda $\psi(x)$ como:
\begin{tcolorbox}[bluebox]
\begin{equation}
  \begin{aligned}
    W(x,p) = \frac{1}{\pi\hbar} \int_{-\infty}^{\infty} \psi^*\left(x + y\right) \psi\left(x - y\right) e^{2ipy/\hbar} \, dy
  \end{aligned}
\end{equation}
\end{tcolorbox}
\comentario{Esta función combina información de posición y momento, pero puede adoptar valores negativos debido a la interferencia cuántica.}
\\La función de Wigner está normalizada de modo que:
\begin{tcolorbox}[bluebox]
\begin{equation}
  \begin{aligned}
    \iint W(x,p) \, dx \, dp = 1
  \end{aligned}
\end{equation}
\end{tcolorbox}
y sus proyecciones reproducen las densidades de probabilidad ordinarias:
\begin{tcolorbox}[bluebox]
\begin{equation}
  \begin{aligned}
    \int W(x,p)\, dp = |\psi(x)|^2, \qquad \int W(x,p)\, dx = |\phi(p)|^2
  \end{aligned}
\end{equation}
\end{tcolorbox}
\comentario{De este modo, la función de Wigner actúa como un puente entre las representaciones de posición y momento.}

\subsection{Propiedades generales}
Entre las propiedades más relevantes de $W(x,p)$ se encuentran:
\begin{enumerate}
  \item Es real, aunque no positiva definida.
  \item Está normalizada a la unidad.
  \item Se transforma covariantemente bajo traslaciones y rotaciones del espacio de fase.
  \item Permite calcular el valor esperado de un observable $\hat{A}$ mediante:
  \begin{tcolorbox}[bluebox]
  \begin{equation}
    \begin{aligned}
      \langle \hat{A} \rangle = \iint W(x,p)\, A_W(x,p)\, dx\,dp
    \end{aligned}
  \end{equation}
  \end{tcolorbox}
  donde $A_W(x,p)$ es el símbolo de Wigner–Weyl del operador $\hat{A}$.
\end{enumerate}

\subsection{Ejemplo: estado coherente en el espacio de fase}
Para un estado coherente $|\alpha\rangle$, la función de Wigner adopta una forma gaussiana centrada en $(x_0,p_0)$:
\begin{tcolorbox}[bluebox]
\begin{equation}
  \begin{aligned}
    W_\alpha(x,p) = \frac{1}{\pi\hbar}\exp\!\left[-\frac{(x-x_0)^2}{\sigma_x^2} - \frac{(p-p_0)^2}{\sigma_p^2}\right]
  \end{aligned}
\end{equation}
\end{tcolorbox}
\comentario{Esta expresión refleja que el estado coherente es el más "clásico" de todos los estados cuánticos, pues su distribución se mantiene mínima y centrada a lo largo del tiempo.}

\subsection{Simetría de paridad y negatividad cuántica}
Las regiones donde $W(x,p)<0$ no poseen interpretación probabilística clásica; representan interferencias puramente cuánticas.  
Esta característica hace de $W(x,p)$ una herramienta esencial para distinguir entre estados cuánticos clásicos y no clásicos.
\comentario{Más adelante se mostrará cómo los estados comprimidos y los estados de Fock de orden alto exhiben regiones de negatividad en $W(x,p)$.}

\subsection{Transformación de Wigner–Weyl}
El formalismo de Wigner puede extenderse a cualquier operador mediante la transformación de Wigner–Weyl:
\begin{tcolorbox}[bluebox]
\begin{equation}
  \begin{aligned}
    A_W(x,p) = \int e^{2ipy/\hbar}\left\langle x - y \middle| \hat{A} \middle| x + y \right\rangle dy
  \end{aligned}
\end{equation}
\end{tcolorbox}
\comentario{Esta transformación asocia a cada operador $\hat{A}$ una función en el espacio de fase, lo que permite expresar la mecánica cuántica en un marco algebraico completamente clásico.}

\section{Estados Comprimidos (Squeezed States)}
Los \textit{estados comprimidos} o \textit{squeezed states} representan una extensión natural de los estados coherentes.  
En ellos, la incertidumbre en una cuadratura se reduce por debajo del límite cuántico estándar, mientras que la incertidumbre conjugada aumenta, de manera que el producto $\Delta X \, \Delta P$ sigue satisfaciendo el principio de Heisenberg.
\comentario{Estos estados son fundamentales en óptica cuántica, interferometría de alta precisión y detección de ondas gravitacionales, donde la reducción de ruido en una cuadratura resulta esencial.}

\subsection{Definición del operador de compresión}
El estado comprimido se genera aplicando el \textit{operador de compresión} $\hat{S}(\zeta)$ sobre el estado de vacío o sobre un estado coherente:
\begin{tcolorbox}[bluebox]
\begin{equation}
  \begin{aligned}
    |\zeta\rangle &= \hat{S}(\zeta)|0\rangle, \\
    \hat{S}(\zeta) &= \exp\!\left[\frac{1}{2}\left(\zeta^* \hat{a}^2 - \zeta (\hat{a}^\dagger)^2\right)\right]
  \end{aligned}
\end{equation}
\end{tcolorbox}
donde $\zeta = r e^{i\theta}$ es un número complejo que determina el grado $r$ y la dirección $\theta$ de la compresión.
\comentario{El operador $\hat{S}(\zeta)$ mezcla los operadores de creación y aniquilación, produciendo correlaciones cuánticas entre ellos.}

\subsection{Efecto sobre los operadores de cuadratura}
El operador de compresión transforma las cuadraturas según:
\begin{tcolorbox}[bluebox]
\begin{equation}
  \begin{aligned}
    \hat{S}^\dagger(\zeta)\,\hat{X}\,\hat{S}(\zeta) &= e^{-r}\hat{X}, \\
    \hat{S}^\dagger(\zeta)\,\hat{P}\,\hat{S}(\zeta) &= e^{r}\hat{P}
  \end{aligned}
\end{equation}
\end{tcolorbox}
\comentario{El parámetro $r$ determina el grado de compresión: una cuadratura se “estrecha” y la otra se “ensancha”.}

\subsection{Dispersión de las cuadraturas}
A partir de la transformación anterior, las incertidumbres del estado comprimido resultan:
\begin{tcolorbox}[bluebox]
\begin{equation}
  \begin{aligned}
    (\Delta X)^2 &= \frac{1}{2} e^{-2r}, \\
    (\Delta P)^2 &= \frac{1}{2} e^{2r}
  \end{aligned}
\end{equation}
\end{tcolorbox}
\comentario{El producto $\Delta X \, \Delta P = \tfrac{1}{2}$ se mantiene constante, garantizando la validez del principio de incertidumbre.}

\subsection{Estado comprimido coherente}
Combinando compresión y desplazamiento se obtiene el \textit{estado comprimido coherente}:
\begin{tcolorbox}[bluebox]
\begin{equation}
  \begin{aligned}
    |\alpha, \zeta\rangle &= \hat{S}(\zeta)\,\hat{D}(\alpha)|0\rangle
  \end{aligned}
\end{equation}
\end{tcolorbox}
\comentario{Este tipo de estado posee una distribución elíptica en el espacio de fase: su centro está desplazado, y su forma refleja la compresión en una de las cuadraturas.}

\subsection{Representación en el espacio de fase}
La función de Wigner correspondiente a un estado comprimido es una elipse en el plano $(x,p)$:
\begin{tcolorbox}[bluebox]
\begin{equation}
  \begin{aligned}
    W_{\alpha,\zeta}(x,p) = \frac{1}{\pi\hbar}
    \exp\!\left[-\frac{(x-x_0)^2}{\sigma_x^2} - \frac{(p-p_0)^2}{\sigma_p^2}\right],
  \end{aligned}
\end{equation}
\end{tcolorbox}
con
\begin{tcolorbox}[bluebox]
\begin{equation}
  \begin{aligned}
    \sigma_x^2 = \frac{\hbar}{2m\omega}e^{-2r}, \qquad
    \sigma_p^2 = \frac{\hbar m\omega}{2}e^{2r}.
  \end{aligned}
\end{equation}
\end{tcolorbox}
\comentario{Para $r>0$, la incertidumbre en $x$ disminuye mientras que la de $p$ aumenta, dando lugar a la compresión cuántica.}

\subsection{Interpretación física}
El estado comprimido puede entenderse como una deformación del vacío cuántico.  
En él, las fluctuaciones de una componente del campo se reducen a costa de amplificar las de la componente conjugada.  
Esta propiedad se aprovecha en experimentos de interferometría para reducir el ruido de medición.
\comentario{Los detectores LIGO y Virgo emplean luz comprimida para mejorar la sensibilidad a las ondas gravitacionales, alcanzando niveles de ruido por debajo del límite estándar cuántico.}

\section{Evolución Temporal del Oscilador Armónico Cuántico}
El estudio de la evolución temporal permite comprender cómo los estados cuánticos del oscilador armónico se comportan bajo la acción del operador de evolución unitario.  
En el caso de estados coherentes o comprimidos, esta evolución mantiene la forma de la función de onda, modificando únicamente su fase y posición en el espacio de fase.
\comentario{El oscilador armónico es uno de los pocos sistemas cuánticos cuya evolución temporal puede obtenerse de forma exacta.}

\subsection{Ecuación de Schrödinger dependiente del tiempo}
La dinámica de cualquier estado cuántico $|\psi(t)\rangle$ está gobernada por la ecuación de Schrödinger:
\begin{tcolorbox}[bluebox]
\begin{equation}
  \begin{aligned}
    i\hbar \frac{d}{dt}|\psi(t)\rangle = \hat{H}|\psi(t)\rangle
  \end{aligned}
\end{equation}
\end{tcolorbox}
Para el oscilador armónico,
\begin{tcolorbox}[bluebox]
\begin{equation}
  \begin{aligned}
    \hat{H} = \hbar\omega\left(\hat{a}^\dagger \hat{a} + \frac{1}{2}\right)
  \end{aligned}
\end{equation}
\end{tcolorbox}
La solución formal de esta ecuación se expresa mediante el operador de evolución unitario:
\begin{tcolorbox}[bluebox]
\begin{equation}
  \begin{aligned}
    \hat{U}(t) = e^{-i\hat{H}t/\hbar}
  \end{aligned}
\end{equation}
\end{tcolorbox}
\comentario{El operador $\hat{U}(t)$ preserva la norma del estado, garantizando la conservación de la probabilidad total.}

\subsection{Evolución de los operadores de creación y aniquilación}
En el marco de Heisenberg, los operadores evolucionan en el tiempo según:
\begin{tcolorbox}[bluebox]
\begin{equation}
  \begin{aligned}
    \hat{a}(t) &= e^{i\hat{H}t/\hbar}\, \hat{a}\, e^{-i\hat{H}t/\hbar}, \\
    \hat{a}^\dagger(t) &= e^{i\hat{H}t/\hbar}\, \hat{a}^\dagger\, e^{-i\hat{H}t/\hbar}
  \end{aligned}
\end{equation}
\end{tcolorbox}
Evaluando las conmutaciones se obtiene:
\begin{tcolorbox}[bluebox]
\begin{equation}
  \begin{aligned}
    \hat{a}(t) &= \hat{a}(0)e^{-i\omega t}, \\
    \hat{a}^\dagger(t) &= \hat{a}^\dagger(0)e^{i\omega t}
  \end{aligned}
\end{equation}
\end{tcolorbox}
\comentario{El operador de aniquilación rota en el espacio de fase con frecuencia $\omega$, describiendo un movimiento circular.}

\subsection{Evolución del estado coherente}
Aplicando el operador de evolución sobre un estado coherente inicial $|\alpha(0)\rangle$, se tiene:
\begin{tcolorbox}[bluebox]
\begin{equation}
  \begin{aligned}
    |\alpha(t)\rangle &= \hat{U}(t)|\alpha(0)\rangle = e^{-i\omega t/2}|\alpha(0)e^{-i\omega t}\rangle
  \end{aligned}
\end{equation}
\end{tcolorbox}
\comentario{El estado coherente conserva su forma, pero rota en el espacio de fase siguiendo la trayectoria clásica del oscilador.}

\subsection{Evolución del estado comprimido}
Para un estado comprimido coherente $|\alpha,\zeta\rangle$, la evolución temporal introduce una rotación tanto del desplazamiento como del parámetro de compresión:
\begin{tcolorbox}[bluebox]
\begin{equation}
  \begin{aligned}
    |\alpha,\zeta;t\rangle = \hat{U}(t)|\alpha,\zeta\rangle
    = e^{-i\omega t/2}\, |\alpha e^{-i\omega t}, \zeta e^{-2i\omega t}\rangle
  \end{aligned}
\end{equation}
\end{tcolorbox}
\comentario{El parámetro $\zeta$ rota con el doble de la frecuencia angular, lo que implica una oscilación periódica de la elipse de incertidumbre en el espacio de fase.}

\subsection{Propagador del oscilador armónico}
Otra forma equivalente de estudiar la evolución es mediante el \textit{propagador} $K(x,t;x',0)$, que relaciona la función de onda inicial con la final:
\begin{tcolorbox}[bluebox]
\begin{equation}
  \begin{aligned}
    \psi(x,t) = \int_{-\infty}^{\infty} K(x,t;x',0)\,\psi(x',0)\,dx'
  \end{aligned}
\end{equation}
\end{tcolorbox}
El propagador exacto del oscilador armónico se obtiene evaluando la integral de camino y resulta:
\begin{tcolorbox}[bluebox]
\begin{equation}
  \begin{aligned}
    K(x,t;x',0) &= \sqrt{\frac{m\omega}{2\pi i\hbar\sin(\omega t)}}
    \exp\!\left[\frac{i m\omega}{2\hbar\sin(\omega t)}\left((x^2+x'^2)\cos(\omega t) - 2xx'\right)\right]
  \end{aligned}
\end{equation}
\end{tcolorbox}
\comentario{Este propagador describe cómo la probabilidad se propaga en el tiempo dentro del potencial armónico. Es una herramienta central en formulaciones de la mecánica cuántica basadas en integrales de camino.}

\subsection{Periodicidad y estabilidad del movimiento}
El oscilador armónico cuántico exhibe periodicidad exacta en su evolución: después de un periodo $T = 2\pi/\omega$, todos los estados recuperan su forma inicial salvo una fase global.
\begin{tcolorbox}[bluebox]
\begin{equation}
  \begin{aligned}
    |\psi(T)\rangle = e^{-i\pi}|\psi(0)\rangle
  \end{aligned}
\end{equation}
\end{tcolorbox}
\comentario{La fase adquirida, llamada \textit{fase de Berry dinámica}, carece de efectos observables, pero tiene implicaciones topológicas en sistemas más complejos.}

\section{Función de Wigner y Observables}
La función de Wigner es una herramienta poderosa para analizar la estructura cuántica de un estado, ya que permite representar las propiedades del sistema en el espacio de fase $(x,p)$.  
A través de ella, es posible calcular valores esperados de operadores, visualizar interferencias cuánticas y distinguir entre estados clásicos y no clásicos.
\comentario{Aunque $W(x,p)$ se asemeja a una densidad de probabilidad, su capacidad de adoptar valores negativos revela la naturaleza cuántica subyacente.}

\subsection{Definición general}
Para un operador de densidad $\hat{\rho}$, la función de Wigner se define como:
\begin{tcolorbox}[bluebox]
\begin{equation}
  \begin{aligned}
    W(x,p) = \frac{1}{\pi\hbar} \int_{-\infty}^{\infty}
    \langle x+y | \hat{\rho} | x-y \rangle \, e^{-2ipy/\hbar} \, dy
  \end{aligned}
\end{equation}
\end{tcolorbox}
\comentario{La integral efectúa una transformada de Fourier sobre la diferencia de coordenadas, combinando la información de posición y momento.}

\subsection{Valores esperados en la representación de Wigner}
El valor esperado de un observable $\hat{A}$ puede calcularse mediante su símbolo de Wigner–Weyl $A_W(x,p)$:
\begin{tcolorbox}[bluebox]
\begin{equation}
  \begin{aligned}
    \langle \hat{A} \rangle = \iint W(x,p)\,A_W(x,p)\,dx\,dp
  \end{aligned}
\end{equation}
\end{tcolorbox}
\comentario{Esta expresión traduce la mecánica cuántica a un formalismo algebraico análogo al de la estadística clásica, pero con correcciones no conmutativas.}

\subsection{Ejemplo: Energía media del oscilador}
El símbolo de Wigner–Weyl del Hamiltoniano del oscilador armónico es:
\begin{tcolorbox}[bluebox]
\begin{equation}
  \begin{aligned}
    H_W(x,p) = \frac{p^2}{2m} + \frac{1}{2}m\omega^2x^2
  \end{aligned}
\end{equation}
\end{tcolorbox}
Para un estado coherente $|\alpha\rangle$, el valor esperado de la energía resulta:
\begin{tcolorbox}[bluebox]
\begin{equation}
  \begin{aligned}
    \langle \hat{H} \rangle_\alpha
    &= \iint W_\alpha(x,p)\,H_W(x,p)\,dx\,dp
    = \hbar\omega\left(|\alpha|^2 + \frac{1}{2}\right)
  \end{aligned}
\end{equation}
\end{tcolorbox}
\comentario{Este resultado coincide con la expresión obtenida directamente en el espacio de Hilbert, validando la equivalencia de ambos enfoques.}

\subsection{Criterios de no-clasicidad}
La presencia de regiones negativas en la función de Wigner indica que el estado cuántico no posee una descripción clásica.  
Este comportamiento se observa en los estados comprimidos intensamente o en superposiciones tipo “gato de Schrödinger”.
\begin{tcolorbox}[bluebox]
\begin{equation}
  \begin{aligned}
    W(x,p) < 0 \quad \Rightarrow \quad \text{Estado no clásico.}
  \end{aligned}
\end{equation}
\end{tcolorbox}
\comentario{Las regiones negativas surgen por interferencia cuántica, una propiedad imposible de reproducir con distribuciones probabilísticas reales.}

\subsection{Simetrías y transformaciones}
La función de Wigner obedece reglas sencillas bajo transformaciones lineales:
\begin{tcolorbox}[bluebox]
\begin{equation}
  \begin{aligned}
    W_\alpha(x,p) &\longrightarrow W_\alpha(x - x_0, p - p_0)
  \end{aligned}
\end{equation}
\end{tcolorbox}
\comentario{Un desplazamiento en el espacio de fase corresponde a la acción del operador de desplazamiento $\hat{D}(\alpha)$.}
Bajo una rotación por un ángulo $\theta$:
\begin{tcolorbox}[bluebox]
\begin{equation}
  \begin{aligned}
    W_\alpha(x,p) &\longrightarrow W_\alpha(x\cos\theta + p\sin\theta, -x\sin\theta + p\cos\theta)
  \end{aligned}
\end{equation}
\end{tcolorbox}
\comentario{Esta propiedad refleja la invariancia del formalismo frente a las rotaciones en el espacio de fase, análoga a la simetría del Hamiltoniano del oscilador.}

\subsection{Reconstrucción experimental}
La función de Wigner puede reconstruirse experimentalmente mediante \textit{tomografía cuántica}, en la que se miden las distribuciones de probabilidad de distintas cuadraturas $\hat{X}_\theta$.
\begin{tcolorbox}[bluebox]
\begin{equation}
  \begin{aligned}
    W(x,p) = \frac{1}{2\pi}\int_0^\pi P(X_\theta,\theta)\, e^{iR(\theta)} \, d\theta
  \end{aligned}
\end{equation}
\end{tcolorbox}
\comentario{En la práctica, $P(X_\theta,\theta)$ se obtiene mediante detección homodina, y el conjunto de mediciones permite reconstruir la función completa $W(x,p)$.}

\subsection{Visualización y aplicaciones}
La representación de Wigner se utiliza para analizar:
\begin{itemize}
  \item La coherencia y pureza de estados cuánticos.
  \item Los efectos de decoherencia y ruido térmico.
  \item La evolución temporal de paquetes de onda y campos ópticos.
\end{itemize}
\comentario{El análisis en el espacio de fase proporciona una visión geométrica del comportamiento cuántico, unificando descripciones clásicas y cuánticas.}

\section{Matriz de Covarianza}
La \textit{matriz de covarianza} es una descripción estadística compacta de un estado cuántico gaussiano.  
Contiene toda la información necesaria sobre las incertidumbres y correlaciones de las cuadraturas del campo electromagnético.
\comentario{Para estados gaussianos, la matriz de covarianza es equivalente a conocer por completo la función de Wigner.}

\subsection{Definición general}
Sea el vector de cuadraturas $\hat{\mathbf{R}} = (\hat{X}, \hat{P})^T$.  
La matriz de covarianza se define como:
\begin{tcolorbox}[bluebox]
\begin{equation}
  \begin{aligned}
    \mathbf{V}_{ij} = \frac{1}{2}\langle \hat{R}_i \hat{R}_j + \hat{R}_j \hat{R}_i \rangle
    - \langle \hat{R}_i \rangle \langle \hat{R}_j \rangle
  \end{aligned}
\end{equation}
\end{tcolorbox}
\comentario{El primer término mide las correlaciones simétricas entre cuadraturas, mientras que el segundo elimina las medias.}
Para un estado de un solo modo, la matriz toma la forma:
\begin{tcolorbox}[bluebox]
\begin{equation}
  \begin{aligned}
    \mathbf{V} =
    \begin{pmatrix}
      \langle \Delta \hat{X}^2 \rangle & \frac{1}{2}\langle \{\hat{X},\hat{P}\} \rangle \\
      \frac{1}{2}\langle \{\hat{P},\hat{X}\} \rangle & \langle \Delta \hat{P}^2 \rangle
    \end{pmatrix}
  \end{aligned}
\end{equation}
\end{tcolorbox}
\comentario{La diagonal contiene las varianzas individuales y los términos fuera de la diagonal representan correlaciones entre cuadraturas.}

\subsection{Propiedades básicas}
La matriz de covarianza cumple la desigualdad de Heisenberg:
\begin{tcolorbox}[bluebox]
\begin{equation}
  \begin{aligned}
    \det(\mathbf{V}) \geq \frac{1}{4}
  \end{aligned}
\end{equation}
\end{tcolorbox}
\comentario{Esta condición garantiza que las incertidumbres nunca violen el principio de Heisenberg.}
Para un estado coherente, se tiene:
\begin{tcolorbox}[bluebox]
\begin{equation}
  \begin{aligned}
    \mathbf{V}_{\text{coh}} =
    \frac{1}{2}
    \begin{pmatrix}
      1 & 0 \\
      0 & 1
    \end{pmatrix}
  \end{aligned}
\end{equation}
\end{tcolorbox}
\comentario{La incertidumbre es mínima y simétrica en ambas cuadraturas.}
Para un estado comprimido:
\begin{tcolorbox}[bluebox]
\begin{equation}
  \begin{aligned}
    \mathbf{V}_{\text{sqz}} =
    \frac{1}{2}
    \begin{pmatrix}
      e^{-2r} & 0 \\
      0 & e^{2r}
    \end{pmatrix}
  \end{aligned}
\end{equation}
\end{tcolorbox}
\comentario{La compresión modifica la diagonal de la matriz, reduciendo una varianza y ampliando la otra.}

\subsection{Covarianza de estados mezclados}
Si el sistema está en contacto con un entorno térmico a temperatura $T$, las varianzas se incrementan de acuerdo con la ocupación media $\bar{n}$ del modo:
\begin{tcolorbox}[bluebox]
\begin{equation}
  \begin{aligned}
    \mathbf{V}_{\text{th}} =
    \left(\bar{n}+\frac{1}{2}\right)
    \begin{pmatrix}
      1 & 0 \\
      0 & 1
    \end{pmatrix},
    \qquad \bar{n} = \frac{1}{e^{\hbar\omega/k_BT} - 1}
  \end{aligned}
\end{equation}
\end{tcolorbox}
\comentario{El estado térmico mantiene la simetría entre cuadraturas, pero con incertidumbres ampliadas debido al ruido térmico.}

\subsection{Correlaciones entre modos}
Para un sistema de dos modos, el vector de cuadraturas es $\hat{\mathbf{R}} = (\hat{X}_1, \hat{P}_1, \hat{X}_2, \hat{P}_2)^T$, y la matriz de covarianza adopta la forma por bloques:
\begin{tcolorbox}[bluebox]
\begin{equation}
  \begin{aligned}
    \mathbf{V} =
    \begin{pmatrix}
      \mathbf{A} & \mathbf{C} \\
      \mathbf{C}^T & \mathbf{B}
    \end{pmatrix}
  \end{aligned}
\end{equation}
\end{tcolorbox}
donde $\mathbf{A}$ y $\mathbf{B}$ representan las covarianzas de cada modo y $\mathbf{C}$ describe las correlaciones entre ellos.
\comentario{Las correlaciones cuánticas entre modos son el origen de la \textit{entrelazación gaussiana}.}

\subsection{Interpretación geométrica}
La matriz de covarianza determina la forma y orientación de la elipse de incertidumbre en el espacio de fase.  
Sus autovalores indican las longitudes de los ejes principales, y su determinante mide el “área cuántica” mínima permitida por el principio de incertidumbre.
\comentario{Esta interpretación geométrica permite visualizar la transición entre estados coherentes, comprimidos y térmicos de manera unificada.}

\input{tex/cap11_deteccion_cuantica}
\section{Amplificación Paramétrica y Dinámica del Campo}
La \textit{amplificación paramétrica} es un proceso óptico no lineal en el que un campo fuerte (la bomba) transfiere energía a uno o más modos del campo electromagnético.  
Este fenómeno es la base física de la generación de luz comprimida, de la amplificación cuántica y del entrelazamiento entre modos ópticos.
\comentario{En el dominio cuántico, los procesos paramétricos se modelan mediante interacciones bilineales entre operadores de creación y aniquilación.}

\subsection{Hamiltoniano efectivo}
El Hamiltoniano que describe la amplificación paramétrica de un modo se puede escribir como:
\begin{tcolorbox}[bluebox]
\begin{equation}
  \begin{aligned}
    \hat{H}_{\text{int}} = i\hbar\kappa\left(\hat{a}^{\dagger 2} e^{-2i\omega_p t} - \hat{a}^2 e^{2i\omega_p t}\right)
  \end{aligned}
\end{equation}
\end{tcolorbox}
donde $\kappa$ es una constante de acoplamiento proporcional a la no linealidad del medio y $\omega_p$ la frecuencia del bombeo.
\comentario{El término $\hat{a}^{\dagger 2}$ representa la creación simultánea de dos fotones —base del fenómeno de “down-conversion” paramétrico.}

\subsection{Ecuaciones de Heisenberg para los operadores}
A partir del Hamiltoniano anterior, las ecuaciones de movimiento de los operadores son:
\begin{tcolorbox}[bluebox]
\begin{equation}
  \begin{aligned}
    \frac{d\hat{a}}{dt} &= 2\kappa \hat{a}^\dagger, \\
    \frac{d\hat{a}^\dagger}{dt} &= 2\kappa \hat{a}
  \end{aligned}
\end{equation}
\end{tcolorbox}
cuya solución es:
\begin{tcolorbox}[bluebox]
\begin{equation}
  \begin{aligned}
    \hat{a}(t) &= \hat{a}(0)\cosh(2\kappa t) + \hat{a}^\dagger(0)\sinh(2\kappa t)
  \end{aligned}
\end{equation}
\end{tcolorbox}
\comentario{Esta mezcla lineal de $\hat{a}$ y $\hat{a}^\dagger$ corresponde exactamente a la transformación inducida por el operador de compresión $\hat{S}(\zeta)$.}

\subsection{Generación de estados comprimidos}
Si el campo inicial está en el vacío $|0\rangle$, la acción del Hamiltoniano paramétrico genera un estado comprimido:
\begin{tcolorbox}[bluebox]
\begin{equation}
  \begin{aligned}
    |\psi(t)\rangle = e^{-i\hat{H}_{\text{int}}t/\hbar}|0\rangle = \hat{S}(\zeta)|0\rangle,
    \quad \text{con} \quad \zeta = 2\kappa t
  \end{aligned}
\end{equation}
\end{tcolorbox}
\comentario{El grado de compresión crece linealmente con el tiempo de interacción y con la intensidad del bombeo óptico.}

\subsection{Amplificación paramétrica de dos modos}
En el caso de un proceso de \textit{down-conversion} con dos modos ($\hat{a}_1$, $\hat{a}_2$), el Hamiltoniano efectivo es:
\begin{tcolorbox}[bluebox]
\begin{equation}
  \begin{aligned}
    \hat{H}_{\text{int}} = i\hbar\kappa\left(\hat{a}_1^\dagger \hat{a}_2^\dagger - \hat{a}_1 \hat{a}_2\right)
  \end{aligned}
\end{equation}
\end{tcolorbox}
\comentario{Este tipo de interacción genera pares de fotones correlacionados —base de la generación de entrelazamiento óptico.}
Las ecuaciones de movimiento para ambos modos son:
\begin{tcolorbox}[bluebox]
\begin{equation}
  \begin{aligned}
    \frac{d\hat{a}_1}{dt} &= \kappa \hat{a}_2^\dagger, \\
    \frac{d\hat{a}_2}{dt} &= \kappa \hat{a}_1^\dagger
  \end{aligned}
\end{equation}
\end{tcolorbox}
con soluciones:
\begin{tcolorbox}[bluebox]
\begin{equation}
  \begin{aligned}
    \hat{a}_1(t) &= \hat{a}_1(0)\cosh(\kappa t) + \hat{a}_2^\dagger(0)\sinh(\kappa t), \\
    \hat{a}_2(t) &= \hat{a}_2(0)\cosh(\kappa t) + \hat{a}_1^\dagger(0)\sinh(\kappa t)
  \end{aligned}
\end{equation}
\end{tcolorbox}
\comentario{Estas ecuaciones revelan la aparición de correlaciones cuánticas entre los modos, es decir, entrelazamiento.}

\subsection{Conservación de la energía}
En la amplificación paramétrica, la energía del campo total se conserva mediante la transferencia desde la bomba hacia los modos generados:
\begin{tcolorbox}[bluebox]
\begin{equation}
  \begin{aligned}
    \hbar\omega_p = \hbar\omega_1 + \hbar\omega_2
  \end{aligned}
\end{equation}
\end{tcolorbox}
\comentario{Este principio garantiza la coherencia del proceso y la correlación energética entre los fotones producidos.}

\subsection{Implementaciones experimentales}
Los procesos paramétricos se implementan en cristales no lineales con susceptibilidad óptica $\chi^{(2)}$, tales como KTP, BBO o PPKTP.  
Cuando son bombeados por un láser intenso, estos materiales generan luz comprimida y entrelazada, detectada posteriormente mediante técnicas de interferometría homodina.
\comentario{La amplificación paramétrica degenerada y no degenerada son pilares fundamentales de la óptica cuántica moderna y la metrología de precisión.}

```latex
\section{Implementación Numérica}
El análisis numérico permite visualizar la evolución de los estados cuánticos y comprobar las propiedades teóricas del oscilador armónico, los estados coherentes, comprimidos y sus respectivas funciones de Wigner.  
En esta sección se describen estrategias de simulación en Python y MATLAB, basadas en el cálculo matricial de operadores y la expansión en la base de Fock.
\comentario{El enfoque numérico traduce la teoría en experimentos virtuales que pueden explorarse y visualizarse de manera controlada.}

\subsection{Discretización de la base de Fock}
Para una simulación finita, se trunca la base de Fock a un número $N_{\text{max}}$ de niveles.  
Los operadores de aniquilación y creación se representan como matrices $N_{\text{max}} \times N_{\text{max}}$:
\begin{tcolorbox}[bluebox]
\begin{equation}
  \begin{aligned}
    (\hat{a})_{n,m} &= \sqrt{m}\,\delta_{n,m-1}, \\
    (\hat{a}^\dagger)_{n,m} &= \sqrt{n}\,\delta_{n,m+1}
  \end{aligned}
\end{equation}
\end{tcolorbox}
\comentario{Con esta representación matricial, los operadores y los estados pueden manipularse mediante álgebra lineal estándar.}

\subsection{Construcción del Hamiltoniano}
El Hamiltoniano del oscilador armónico se construye en esta base como:
\begin{tcolorbox}[bluebox]
\begin{equation}
  \begin{aligned}
    \hat{H} = \hbar\omega\left(\hat{a}^\dagger\hat{a} + \frac{1}{2}\mathbb{I}\right)
  \end{aligned}
\end{equation}
\end{tcolorbox}
\comentario{Esta forma matricial permite calcular numéricamente los autovalores y autovectores, verificando la cuantización de la energía.}

\subsection{Evolución temporal numérica}
El operador de evolución unitario se aproxima mediante la exponencial matricial:
\begin{tcolorbox}[bluebox]
\begin{equation}
  \begin{aligned}
    \hat{U}(t) = e^{-i\hat{H}t/\hbar}
  \end{aligned}
\end{equation}
\end{tcolorbox}
En Python (usando \texttt{NumPy} y \texttt{SciPy}), se implementa como:
\begin{lstlisting}[language=Python]
import numpy as np
from scipy.linalg import expm

U_t = expm(-1j * H * t / hbar)
psi_t = U_t @ psi0
\end{lstlisting}
\comentario{Esta aproximación permite estudiar la rotación temporal de un estado coherente o comprimido en el espacio de fase.}

\subsection{Cálculo de la función de Wigner}
La función de Wigner puede calcularse numéricamente mediante su definición integral o usando librerías dedicadas.
Para un estado $\rho = |\psi\rangle\langle\psi|$, la versión discreta se implementa como:
\begin{tcolorbox}[bluebox]
\begin{equation}
    \begin{aligned}
        W(x,p) = \frac{1}{\pi\hbar}\sum_{m,n}\rho_{mn},e^{-2i p (x_m - x_n)/\hbar}
    \end{aligned}
\end{equation}
\end{tcolorbox}
Ejemplo en Python (usando \texttt{QuTiP}):
\begin{lstlisting}[language=Python]
from qutip import coherent, wigner, Qobj
import matplotlib.pyplot as plt

alpha = 1.0
psi = coherent(40, alpha)
x = np.linspace(-5, 5, 200)
p = np.linspace(-5, 5, 200)
W = wigner(psi, x, p)

plt.contourf(x, p, W, 100, cmap='RdBu_r')
plt.xlabel("x")
plt.ylabel("p")

plt.show()
\end{lstlisting}
\comentario{El resultado visualiza la distribución gaussiana en el espacio de fase y su evolución en el tiempo.}

\subsection{Simulación de compresión}
La acción del operador de compresión se modela aplicando la transformación:
\begin{tcolorbox}[bluebox]
\begin{equation}
    \begin{aligned}
        \hat{S}(\zeta) = \exp!\left[\frac{1}{2}(\zeta^*\hat{a}^2 - \zeta(\hat{a}^\dagger)^2)\right]
    \end{aligned}
\end{equation}
\end{tcolorbox}
Ejemplo en \texttt{QuTiP}:
\begin{lstlisting}[language=Python]
from qutip import squeeze, expect, sigmax

r = 0.8
zeta = r * np.exp(1j * 0)
S = squeeze(40, zeta)
psi_sqz = S * psi
\end{lstlisting}
\comentario{Este código genera un estado comprimido con factor $r=0.8$ y permite comparar las incertidumbres $\Delta X$ y $\Delta P$.}

\subsection{Representación de la matriz de covarianza}
Numéricamente, la matriz de covarianza se calcula a partir de los valores esperados de las cuadraturas:
\begin{tcolorbox}[bluebox]
\begin{equation}
    \begin{aligned}
        V_{ij} = \frac{1}{2}\langle \hat{R}_i \hat{R}_j + \hat{R}_j \hat{R}_i \rangle - \langle \hat{R}_i \rangle \langle \hat{R}_j \rangle
    \end{aligned}
\end{equation}
\end{tcolorbox}
Ejemplo en Python:
\begin{lstlisting}[language=Python]
from qutip import quadrature

X = quadrature(psi_sqz, 0)
P = quadrature(psi_sqz, np.pi/2)
V = np.array([[np.var(X), 0], [0, np.var(P)]])
\end{lstlisting}
\comentario{El resultado numérico confirma las relaciones teóricas: $(\Delta X)^2 = \tfrac{1}{2}e^{-2r}$ y $(\Delta P)^2 = \tfrac{1}{2}e^{2r}$.}

\appendix
\section*{Apéndice A: Ecuaciones Fundamentales}
\addcontentsline{toc}{section}{Apéndice A: Ecuaciones Fundamentales}

Este apéndice reúne las ecuaciones clave que estructuran la teoría del oscilador armónico cuántico y sus extensiones hacia los estados coherentes, comprimidos y correlacionados.  
Cada expresión se presenta con su numeración independiente para referencia rápida.

\renewcommand{\theequation}{A.\arabic{equation}}
\setcounter{equation}{0}

%--------------------------------------------------------
\subsection*{A.1. Ecuación de movimiento clásica}
\begin{tcolorbox}[bluebox]
\begin{equation}
  \begin{aligned}
    m\frac{d^2x}{dt^2} + kx = 0
  \end{aligned}
\end{equation}
\end{tcolorbox}
\comentario{Describe la oscilación armónica clásica con frecuencia angular $\omega = \sqrt{k/m}$.}

%--------------------------------------------------------
\subsection*{A.2. Hamiltoniano clásico y cuántico}
\begin{tcolorbox}[bluebox]
\begin{equation}
  \begin{aligned}
    \hat{H} = \frac{\hat{p}^2}{2m} + \frac{1}{2}m\omega^2\hat{x}^2
  \end{aligned}
\end{equation}
\end{tcolorbox}
\comentario{Base del oscilador armónico cuántico, obtenida por promoción de $x,p$ a operadores.}

%--------------------------------------------------------
\subsection*{A.3. Operadores de creación y aniquilación}
\begin{tcolorbox}[bluebox]
\begin{equation}
  \begin{aligned}
    \hat{a} = \sqrt{\frac{m\omega}{2\hbar}}\left(\hat{x} + \frac{i}{m\omega}\hat{p}\right), \quad
    \hat{a}^\dagger = \sqrt{\frac{m\omega}{2\hbar}}\left(\hat{x} - \frac{i}{m\omega}\hat{p}\right)
  \end{aligned}
\end{equation}
\end{tcolorbox}
\comentario{Definen la estructura algebraica del oscilador; satisfacen $[\hat{a},\hat{a}^\dagger]=1$.}

%--------------------------------------------------------
\subsection*{A.4. Energía cuantizada}
\begin{tcolorbox}[bluebox]
\begin{equation}
  \begin{aligned}
    E_n = \hbar\omega\left(n + \frac{1}{2}\right)
  \end{aligned}
\end{equation}
\end{tcolorbox}
\comentario{Autovalores del Hamiltoniano del oscilador: cuantización discreta de la energía.}

%--------------------------------------------------------
\subsection*{A.5. Estado coherente}
\begin{tcolorbox}[bluebox]
\begin{equation}
  \begin{aligned}
    |\alpha\rangle = e^{-|\alpha|^2/2}\sum_{n=0}^\infty \frac{\alpha^n}{\sqrt{n!}}|n\rangle
  \end{aligned}
\end{equation}
\end{tcolorbox}
\comentario{Solución del autovalor $\hat{a}|\alpha\rangle = \alpha|\alpha\rangle$; reproduce el movimiento clásico.}

%--------------------------------------------------------
\subsection*{A.6. Operador de desplazamiento}
\begin{tcolorbox}[bluebox]
\begin{equation}
  \begin{aligned}
    \hat{D}(\alpha) = e^{\alpha\hat{a}^\dagger - \alpha^*\hat{a}}
  \end{aligned}
\end{equation}
\end{tcolorbox}
\comentario{Desplaza el estado de vacío en el espacio de fase: $|\alpha\rangle = \hat{D}(\alpha)|0\rangle$.}

%--------------------------------------------------------
\subsection*{A.7. Operador de compresión (squeezing)}
\begin{tcolorbox}[bluebox]
\begin{equation}
  \begin{aligned}
    \hat{S}(\zeta) = \exp\!\left[\frac{1}{2}\left(\zeta^*\hat{a}^2 - \zeta(\hat{a}^\dagger)^2\right)\right]
  \end{aligned}
\end{equation}
\end{tcolorbox}
\comentario{Reduce la incertidumbre en una cuadratura a costa de aumentar la de la conjugada.}

%--------------------------------------------------------
\subsection*{A.8. Función de Wigner}
\begin{tcolorbox}[bluebox]
\begin{equation}
  \begin{aligned}
    W(x,p) = \frac{1}{\pi\hbar}\int_{-\infty}^{\infty}
    \psi^*\!\left(x+y\right)\psi\!\left(x-y\right)e^{2ipy/\hbar}\,dy
  \end{aligned}
\end{equation}
\end{tcolorbox}
\comentario{Representación cuasi-probabilística del estado cuántico en el espacio de fase.}

%--------------------------------------------------------
\subsection*{A.9. Matriz de covarianza}
\begin{tcolorbox}[bluebox]
\begin{equation}
  \begin{aligned}
    V_{ij} = \frac{1}{2}\langle \hat{R}_i\hat{R}_j + \hat{R}_j\hat{R}_i\rangle
    - \langle \hat{R}_i\rangle\langle \hat{R}_j\rangle
  \end{aligned}
\end{equation}
\end{tcolorbox}
\comentario{Caracteriza las correlaciones entre cuadraturas y la incertidumbre cuántica mínima.}

%--------------------------------------------------------
\subsection*{A.10. Hamiltoniano paramétrico}
\begin{tcolorbox}[bluebox]
\begin{equation}
  \begin{aligned}
    \hat{H}_{\text{int}} = i\hbar\kappa\left(\hat{a}^{\dagger 2} - \hat{a}^2\right)
  \end{aligned}
\end{equation}
\end{tcolorbox}
\comentario{Modelo idealizado de la amplificación paramétrica responsable de generar luz comprimida.}

%--------------------------------------------------------
\subsection*{A.11. Condición de incertidumbre de Heisenberg}
\begin{tcolorbox}[bluebox]
\begin{equation}
  \begin{aligned}
    \det(\mathbf{V}) \geq \frac{1}{4}
  \end{aligned}
\end{equation}
\end{tcolorbox}
\comentario{Forma matricial general de la desigualdad de Heisenberg, válida para todo estado gaussiano.}


\end{document}