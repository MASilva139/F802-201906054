\section{Estados Comprimidos (Squeezed States)}
Los \textit{estados comprimidos} o \textit{squeezed states} representan una extensión natural de los estados coherentes.  
En ellos, la incertidumbre en una cuadratura se reduce por debajo del límite cuántico estándar, mientras que la incertidumbre conjugada aumenta, de manera que el producto $\Delta X \, \Delta P$ sigue satisfaciendo el principio de Heisenberg.
\comentario{Estos estados son fundamentales en óptica cuántica, interferometría de alta precisión y detección de ondas gravitacionales, donde la reducción de ruido en una cuadratura resulta esencial.}

\subsection{Definición del operador de compresión}
El estado comprimido se genera aplicando el \textit{operador de compresión} $\hat{S}(\zeta)$ sobre el estado de vacío o sobre un estado coherente:
\begin{tcolorbox}[bluebox]
\begin{equation}
  \begin{aligned}
    |\zeta\rangle &= \hat{S}(\zeta)|0\rangle, \\
    \hat{S}(\zeta) &= \exp\!\left[\frac{1}{2}\left(\zeta^* \hat{a}^2 - \zeta (\hat{a}^\dagger)^2\right)\right]
  \end{aligned}
\end{equation}
\end{tcolorbox}
donde $\zeta = r e^{i\theta}$ es un número complejo que determina el grado $r$ y la dirección $\theta$ de la compresión.
\comentario{El operador $\hat{S}(\zeta)$ mezcla los operadores de creación y aniquilación, produciendo correlaciones cuánticas entre ellos.}

\subsection{Efecto sobre los operadores de cuadratura}
El operador de compresión transforma las cuadraturas según:
\begin{tcolorbox}[bluebox]
\begin{equation}
  \begin{aligned}
    \hat{S}^\dagger(\zeta)\,\hat{X}\,\hat{S}(\zeta) &= e^{-r}\hat{X}, \\
    \hat{S}^\dagger(\zeta)\,\hat{P}\,\hat{S}(\zeta) &= e^{r}\hat{P}
  \end{aligned}
\end{equation}
\end{tcolorbox}
\comentario{El parámetro $r$ determina el grado de compresión: una cuadratura se “estrecha” y la otra se “ensancha”.}

\subsection{Dispersión de las cuadraturas}
A partir de la transformación anterior, las incertidumbres del estado comprimido resultan:
\begin{tcolorbox}[bluebox]
\begin{equation}
  \begin{aligned}
    (\Delta X)^2 &= \frac{1}{2} e^{-2r}, \\
    (\Delta P)^2 &= \frac{1}{2} e^{2r}
  \end{aligned}
\end{equation}
\end{tcolorbox}
\comentario{El producto $\Delta X \, \Delta P = \tfrac{1}{2}$ se mantiene constante, garantizando la validez del principio de incertidumbre.}

\subsection{Estado comprimido coherente}
Combinando compresión y desplazamiento se obtiene el \textit{estado comprimido coherente}:
\begin{tcolorbox}[bluebox]
\begin{equation}
  \begin{aligned}
    |\alpha, \zeta\rangle &= \hat{S}(\zeta)\,\hat{D}(\alpha)|0\rangle
  \end{aligned}
\end{equation}
\end{tcolorbox}
\comentario{Este tipo de estado posee una distribución elíptica en el espacio de fase: su centro está desplazado, y su forma refleja la compresión en una de las cuadraturas.}

\subsection{Representación en el espacio de fase}
La función de Wigner correspondiente a un estado comprimido es una elipse en el plano $(x,p)$:
\begin{tcolorbox}[bluebox]
\begin{equation}
  \begin{aligned}
    W_{\alpha,\zeta}(x,p) = \frac{1}{\pi\hbar}
    \exp\!\left[-\frac{(x-x_0)^2}{\sigma_x^2} - \frac{(p-p_0)^2}{\sigma_p^2}\right],
  \end{aligned}
\end{equation}
\end{tcolorbox}
con
\begin{tcolorbox}[bluebox]
\begin{equation}
  \begin{aligned}
    \sigma_x^2 = \frac{\hbar}{2m\omega}e^{-2r}, \qquad
    \sigma_p^2 = \frac{\hbar m\omega}{2}e^{2r}.
  \end{aligned}
\end{equation}
\end{tcolorbox}
\comentario{Para $r>0$, la incertidumbre en $x$ disminuye mientras que la de $p$ aumenta, dando lugar a la compresión cuántica.}

\subsection{Interpretación física}
El estado comprimido puede entenderse como una deformación del vacío cuántico.  
En él, las fluctuaciones de una componente del campo se reducen a costa de amplificar las de la componente conjugada.  
Esta propiedad se aprovecha en experimentos de interferometría para reducir el ruido de medición.
\comentario{Los detectores LIGO y Virgo emplean luz comprimida para mejorar la sensibilidad a las ondas gravitacionales, alcanzando niveles de ruido por debajo del límite estándar cuántico.}
