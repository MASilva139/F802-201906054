\section{Estados Coherentes}
Los \textit{estados coherentes} son soluciones especiales del oscilador armónico cuántico que reproducen, en el límite cuántico, el comportamiento de una oscilación clásica.  
Fueron introducidos formalmente por Roy J. Glauber en la década de 1960 como la base teórica de la \textit{óptica cuántica} moderna.
\comentario{Un estado coherente puede considerarse el análogo cuántico de una onda clásica monocromática.}

\subsection{Definición del estado coherente}
El estado coherente $|\alpha\rangle$ se define como un \textit{autovector del operador de aniquilación} $\hat{a}$:
\begin{tcolorbox}[bluebox]
\begin{equation}
  \begin{aligned}
    \hat{a}|\alpha\rangle = \alpha |\alpha\rangle
  \end{aligned}
\end{equation}
\end{tcolorbox}
donde $\alpha$ es un número complejo que codifica tanto la amplitud como la fase del estado.
\comentario{A diferencia de los estados de número $|n\rangle$, los estados coherentes no tienen un número definido de cuantos, sino una distribución de Poisson en energía.}
La expansión de $|\alpha\rangle$ en la base de Fock se obtiene aplicando el operador de desplazamiento:
\begin{tcolorbox}[bluebox]
\begin{equation}
  \begin{aligned}
    |\alpha\rangle = e^{-|\alpha|^2/2}\sum_{n=0}^{\infty}\frac{\alpha^n}{\sqrt{n!}}|n\rangle
  \end{aligned}
\end{equation}
\end{tcolorbox}
\comentario{El coeficiente exponencial garantiza la normalización $\langle \alpha|\alpha\rangle = 1$.}

\subsection{Propiedades principales}
Los estados coherentes poseen propiedades análogas a las de una onda clásica:
\begin{enumerate}
  \item Minimiza el principio de incertidumbre de Heisenberg:
  \begin{tcolorbox}[bluebox]
  \begin{equation}
    \begin{aligned}
      \Delta x\,\Delta p = \frac{\hbar}{2}
    \end{aligned}
  \end{equation}
  \end{tcolorbox}
  \comentario{Esto significa que las incertidumbres en posición y momento están igualmente repartidas.}
  \item La media de los operadores $\hat{x}$ y $\hat{p}$ evoluciona como en el caso clásico:
  \begin{tcolorbox}[bluebox]
  \begin{equation}
    \begin{aligned}
      \langle \hat{x}(t) \rangle &= \sqrt{\frac{2\hbar}{m\omega}}\Re[\alpha e^{-i\omega t}], \\
      \langle \hat{p}(t) \rangle &= \sqrt{2\hbar m\omega}\Im[\alpha e^{-i\omega t}]
    \end{aligned}
  \end{equation}
  \end{tcolorbox}
  \comentario{Ambas expectativas oscilan con frecuencia $\omega$, reproduciendo el movimiento clásico.}
\end{enumerate}

\subsection{Operador de desplazamiento}
Otra forma equivalente de construir $|\alpha\rangle$ es mediante el operador de desplazamiento:
\begin{tcolorbox}[bluebox]
\begin{equation}
  \begin{aligned}
    \hat{D}(\alpha) &= e^{\alpha\hat{a}^\dagger - \alpha^*\hat{a}}, \\
    |\alpha\rangle &= \hat{D}(\alpha)|0\rangle
  \end{aligned}
\end{equation}
\end{tcolorbox}
\comentario{El operador $\hat{D}(\alpha)$ desplaza el estado de vacío en el espacio de fase, generando el estado coherente correspondiente.}

\subsection{Interpretación en el espacio de fase}
En la representación de Wigner, el estado coherente aparece como una distribución gaussiana centrada en el punto $(x_0, p_0)$ del espacio de fase:
\begin{tcolorbox}[bluebox]
\begin{equation}
  \begin{aligned}
    W_\alpha(x,p) = \frac{1}{\pi\hbar}\exp\left[-\frac{(x-x_0)^2}{\sigma_x^2} - \frac{(p-p_0)^2}{\sigma_p^2}\right]
  \end{aligned}
\end{equation}
\end{tcolorbox}
\comentario{Esta forma resalta la dualidad del estado coherente: una función cuántica que conserva la localización mínima permitida por el principio de incertidumbre.}

\subsection{Energía media y dispersión}
El valor medio de la energía en el estado $|\alpha\rangle$ es:
\begin{tcolorbox}[bluebox]
\begin{equation}
  \begin{aligned}
    \langle \hat{H} \rangle_\alpha &= \hbar\omega\left(|\alpha|^2 + \frac{1}{2}\right)
  \end{aligned}
\end{equation}
\end{tcolorbox}
\comentario{La energía media se comporta como en un oscilador clásico, con un término adicional correspondiente a la energía del punto cero.}
