\section{Oscilador Armónico Clásico}
El oscilador armónico clásico describe el movimiento de una partícula sujeta a una fuerza restauradora proporcional a su desplazamiento. Matemáticamente, esta relación se expresa mediante la \textit{segunda ley de Newton}:
\begin{tcolorbox}[bluebox]
\begin{equation}
  \begin{aligned}
    F &= -k\,x
  \end{aligned}
\end{equation}
\end{tcolorbox}
\comentario{La constante $k$ caracteriza la rigidez del sistema. Cuanto mayor sea $k$, más intensa será la fuerza de restitución.}
Aplicando la relación de Newton $F = m\,a$, obtenemos la ecuación diferencial del movimiento:
\begin{tcolorbox}[bluebox]
\begin{equation}
  \begin{aligned}
    m\frac{d^2x}{dt^2} + kx = 0
  \end{aligned}
\end{equation}
\end{tcolorbox}
Esta ecuación tiene como solución general:
\begin{tcolorbox}[bluebox]
\begin{equation}
  \begin{aligned}
    x(t) = A\cos(\omega t + \phi)
  \end{aligned}
\end{equation}
\end{tcolorbox}
donde $\omega = \sqrt{k/m}$ es la \textit{frecuencia angular}, $A$ la \textit{amplitud} y $\phi$ la \textit{fase inicial}.
\comentario{El movimiento es periódico, y la energía total del sistema se mantiene constante.}
\\La energía total se expresa mediante el Hamiltoniano clásico:
\begin{tcolorbox}[bluebox]
\begin{equation}
  \begin{aligned}
    H &= \frac{p^2}{2m} + \frac{1}{2}m\omega^2x^2
  \end{aligned}
\end{equation}
\end{tcolorbox}
donde $p = m\,\dot{x}$ es el momento lineal. Este Hamiltoniano servirá como punto de partida para la formulación cuántica.
