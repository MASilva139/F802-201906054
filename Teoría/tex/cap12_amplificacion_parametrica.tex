\section{Amplificación Paramétrica y Dinámica del Campo}
La \textit{amplificación paramétrica} es un proceso óptico no lineal en el que un campo fuerte (la bomba) transfiere energía a uno o más modos del campo electromagnético.  
Este fenómeno es la base física de la generación de luz comprimida, de la amplificación cuántica y del entrelazamiento entre modos ópticos.
\comentario{En el dominio cuántico, los procesos paramétricos se modelan mediante interacciones bilineales entre operadores de creación y aniquilación.}

\subsection{Hamiltoniano efectivo}
El Hamiltoniano que describe la amplificación paramétrica de un modo se puede escribir como:
\begin{tcolorbox}[bluebox]
\begin{equation}
  \begin{aligned}
    \hat{H}_{\text{int}} = i\hbar\kappa\left(\hat{a}^{\dagger 2} e^{-2i\omega_p t} - \hat{a}^2 e^{2i\omega_p t}\right)
  \end{aligned}
\end{equation}
\end{tcolorbox}
donde $\kappa$ es una constante de acoplamiento proporcional a la no linealidad del medio y $\omega_p$ la frecuencia del bombeo.
\comentario{El término $\hat{a}^{\dagger 2}$ representa la creación simultánea de dos fotones —base del fenómeno de “down-conversion” paramétrico.}

\subsection{Ecuaciones de Heisenberg para los operadores}
A partir del Hamiltoniano anterior, las ecuaciones de movimiento de los operadores son:
\begin{tcolorbox}[bluebox]
\begin{equation}
  \begin{aligned}
    \frac{d\hat{a}}{dt} &= 2\kappa \hat{a}^\dagger, \\
    \frac{d\hat{a}^\dagger}{dt} &= 2\kappa \hat{a}
  \end{aligned}
\end{equation}
\end{tcolorbox}
cuya solución es:
\begin{tcolorbox}[bluebox]
\begin{equation}
  \begin{aligned}
    \hat{a}(t) &= \hat{a}(0)\cosh(2\kappa t) + \hat{a}^\dagger(0)\sinh(2\kappa t)
  \end{aligned}
\end{equation}
\end{tcolorbox}
\comentario{Esta mezcla lineal de $\hat{a}$ y $\hat{a}^\dagger$ corresponde exactamente a la transformación inducida por el operador de compresión $\hat{S}(\zeta)$.}

\subsection{Generación de estados comprimidos}
Si el campo inicial está en el vacío $|0\rangle$, la acción del Hamiltoniano paramétrico genera un estado comprimido:
\begin{tcolorbox}[bluebox]
\begin{equation}
  \begin{aligned}
    |\psi(t)\rangle = e^{-i\hat{H}_{\text{int}}t/\hbar}|0\rangle = \hat{S}(\zeta)|0\rangle,
    \quad \text{con} \quad \zeta = 2\kappa t
  \end{aligned}
\end{equation}
\end{tcolorbox}
\comentario{El grado de compresión crece linealmente con el tiempo de interacción y con la intensidad del bombeo óptico.}

\subsection{Amplificación paramétrica de dos modos}
En el caso de un proceso de \textit{down-conversion} con dos modos ($\hat{a}_1$, $\hat{a}_2$), el Hamiltoniano efectivo es:
\begin{tcolorbox}[bluebox]
\begin{equation}
  \begin{aligned}
    \hat{H}_{\text{int}} = i\hbar\kappa\left(\hat{a}_1^\dagger \hat{a}_2^\dagger - \hat{a}_1 \hat{a}_2\right)
  \end{aligned}
\end{equation}
\end{tcolorbox}
\comentario{Este tipo de interacción genera pares de fotones correlacionados —base de la generación de entrelazamiento óptico.}
Las ecuaciones de movimiento para ambos modos son:
\begin{tcolorbox}[bluebox]
\begin{equation}
  \begin{aligned}
    \frac{d\hat{a}_1}{dt} &= \kappa \hat{a}_2^\dagger, \\
    \frac{d\hat{a}_2}{dt} &= \kappa \hat{a}_1^\dagger
  \end{aligned}
\end{equation}
\end{tcolorbox}
con soluciones:
\begin{tcolorbox}[bluebox]
\begin{equation}
  \begin{aligned}
    \hat{a}_1(t) &= \hat{a}_1(0)\cosh(\kappa t) + \hat{a}_2^\dagger(0)\sinh(\kappa t), \\
    \hat{a}_2(t) &= \hat{a}_2(0)\cosh(\kappa t) + \hat{a}_1^\dagger(0)\sinh(\kappa t)
  \end{aligned}
\end{equation}
\end{tcolorbox}
\comentario{Estas ecuaciones revelan la aparición de correlaciones cuánticas entre los modos, es decir, entrelazamiento.}

\subsection{Conservación de la energía}
En la amplificación paramétrica, la energía del campo total se conserva mediante la transferencia desde la bomba hacia los modos generados:
\begin{tcolorbox}[bluebox]
\begin{equation}
  \begin{aligned}
    \hbar\omega_p = \hbar\omega_1 + \hbar\omega_2
  \end{aligned}
\end{equation}
\end{tcolorbox}
\comentario{Este principio garantiza la coherencia del proceso y la correlación energética entre los fotones producidos.}

\subsection{Implementaciones experimentales}
Los procesos paramétricos se implementan en cristales no lineales con susceptibilidad óptica $\chi^{(2)}$, tales como KTP, BBO o PPKTP.  
Cuando son bombeados por un láser intenso, estos materiales generan luz comprimida y entrelazada, detectada posteriormente mediante técnicas de interferometría homodina.
\comentario{La amplificación paramétrica degenerada y no degenerada son pilares fundamentales de la óptica cuántica moderna y la metrología de precisión.}
