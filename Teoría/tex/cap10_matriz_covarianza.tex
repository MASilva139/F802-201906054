\section{Matriz de Covarianza}
La \textit{matriz de covarianza} es una descripción estadística compacta de un estado cuántico gaussiano.  
Contiene toda la información necesaria sobre las incertidumbres y correlaciones de las cuadraturas del campo electromagnético.
\comentario{Para estados gaussianos, la matriz de covarianza es equivalente a conocer por completo la función de Wigner.}

\subsection{Definición general}
Sea el vector de cuadraturas $\hat{\mathbf{R}} = (\hat{X}, \hat{P})^T$.  
La matriz de covarianza se define como:
\begin{tcolorbox}[bluebox]
\begin{equation}
  \begin{aligned}
    \mathbf{V}_{ij} = \frac{1}{2}\langle \hat{R}_i \hat{R}_j + \hat{R}_j \hat{R}_i \rangle
    - \langle \hat{R}_i \rangle \langle \hat{R}_j \rangle
  \end{aligned}
\end{equation}
\end{tcolorbox}
\comentario{El primer término mide las correlaciones simétricas entre cuadraturas, mientras que el segundo elimina las medias.}
Para un estado de un solo modo, la matriz toma la forma:
\begin{tcolorbox}[bluebox]
\begin{equation}
  \begin{aligned}
    \mathbf{V} =
    \begin{pmatrix}
      \langle \Delta \hat{X}^2 \rangle & \frac{1}{2}\langle \{\hat{X},\hat{P}\} \rangle \\
      \frac{1}{2}\langle \{\hat{P},\hat{X}\} \rangle & \langle \Delta \hat{P}^2 \rangle
    \end{pmatrix}
  \end{aligned}
\end{equation}
\end{tcolorbox}
\comentario{La diagonal contiene las varianzas individuales y los términos fuera de la diagonal representan correlaciones entre cuadraturas.}

\subsection{Propiedades básicas}
La matriz de covarianza cumple la desigualdad de Heisenberg:
\begin{tcolorbox}[bluebox]
\begin{equation}
  \begin{aligned}
    \det(\mathbf{V}) \geq \frac{1}{4}
  \end{aligned}
\end{equation}
\end{tcolorbox}
\comentario{Esta condición garantiza que las incertidumbres nunca violen el principio de Heisenberg.}
Para un estado coherente, se tiene:
\begin{tcolorbox}[bluebox]
\begin{equation}
  \begin{aligned}
    \mathbf{V}_{\text{coh}} =
    \frac{1}{2}
    \begin{pmatrix}
      1 & 0 \\
      0 & 1
    \end{pmatrix}
  \end{aligned}
\end{equation}
\end{tcolorbox}
\comentario{La incertidumbre es mínima y simétrica en ambas cuadraturas.}
Para un estado comprimido:
\begin{tcolorbox}[bluebox]
\begin{equation}
  \begin{aligned}
    \mathbf{V}_{\text{sqz}} =
    \frac{1}{2}
    \begin{pmatrix}
      e^{-2r} & 0 \\
      0 & e^{2r}
    \end{pmatrix}
  \end{aligned}
\end{equation}
\end{tcolorbox}
\comentario{La compresión modifica la diagonal de la matriz, reduciendo una varianza y ampliando la otra.}

\subsection{Covarianza de estados mezclados}
Si el sistema está en contacto con un entorno térmico a temperatura $T$, las varianzas se incrementan de acuerdo con la ocupación media $\bar{n}$ del modo:
\begin{tcolorbox}[bluebox]
\begin{equation}
  \begin{aligned}
    \mathbf{V}_{\text{th}} =
    \left(\bar{n}+\frac{1}{2}\right)
    \begin{pmatrix}
      1 & 0 \\
      0 & 1
    \end{pmatrix},
    \qquad \bar{n} = \frac{1}{e^{\hbar\omega/k_BT} - 1}
  \end{aligned}
\end{equation}
\end{tcolorbox}
\comentario{El estado térmico mantiene la simetría entre cuadraturas, pero con incertidumbres ampliadas debido al ruido térmico.}

\subsection{Correlaciones entre modos}
Para un sistema de dos modos, el vector de cuadraturas es $\hat{\mathbf{R}} = (\hat{X}_1, \hat{P}_1, \hat{X}_2, \hat{P}_2)^T$, y la matriz de covarianza adopta la forma por bloques:
\begin{tcolorbox}[bluebox]
\begin{equation}
  \begin{aligned}
    \mathbf{V} =
    \begin{pmatrix}
      \mathbf{A} & \mathbf{C} \\
      \mathbf{C}^T & \mathbf{B}
    \end{pmatrix}
  \end{aligned}
\end{equation}
\end{tcolorbox}
donde $\mathbf{A}$ y $\mathbf{B}$ representan las covarianzas de cada modo y $\mathbf{C}$ describe las correlaciones entre ellos.
\comentario{Las correlaciones cuánticas entre modos son el origen de la \textit{entrelazación gaussiana}.}

\subsection{Interpretación geométrica}
La matriz de covarianza determina la forma y orientación de la elipse de incertidumbre en el espacio de fase.  
Sus autovalores indican las longitudes de los ejes principales, y su determinante mide el “área cuántica” mínima permitida por el principio de incertidumbre.
\comentario{Esta interpretación geométrica permite visualizar la transición entre estados coherentes, comprimidos y térmicos de manera unificada.}
