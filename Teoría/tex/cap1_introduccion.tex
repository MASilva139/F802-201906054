\section{Introducción}
El oscilador armónico es uno de los sistemas más fundamentales tanto en física clásica como en mecánica cuántica. Su análisis permite comprender el comportamiento de una amplia variedad de sistemas físicos: desde la vibración de un átomo en una molécula hasta el modo de un campo electromagnético confinado en una cavidad óptica.
\comentario{En esta sección se presenta el contexto histórico y la motivación del estudio.}
\begin{tcolorbox}[bluebox]
\begin{equation}
  \begin{aligned}
    H &= \frac{p^2}{2m} + \frac{1}{2}m\omega^2x^2
  \end{aligned}
\end{equation}
\end{tcolorbox}
\comentario{Esta ecuación representa el Hamiltoniano clásico del oscilador armónico. El primer término corresponde a la energía cinética y el segundo a la energía potencial.}
Para pasar al régimen cuántico, se promueven las variables $x$ y $p$ a operadores, de modo que:
\begin{tcolorbox}[bluebox]
\begin{equation}
  \begin{aligned}
    \hat{x} &= x, \qquad \hat{p} = -i\hbar \frac{\partial}{\partial x}
  \end{aligned}
\end{equation}
\end{tcolorbox}
\comentario{De esta manera, el Hamiltoniano cuántico toma la forma}
\begin{tcolorbox}[bluebox]
\begin{equation}
  \begin{aligned}
    \hat{H} = \frac{\hat{p}^2}{2m} + \frac{1}{2}m\omega^2\hat{x}^2
  \end{aligned}
\end{equation}
\end{tcolorbox}
