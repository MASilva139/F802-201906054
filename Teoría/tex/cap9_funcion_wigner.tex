\section{Función de Wigner y Observables}
La función de Wigner es una herramienta poderosa para analizar la estructura cuántica de un estado, ya que permite representar las propiedades del sistema en el espacio de fase $(x,p)$.  
A través de ella, es posible calcular valores esperados de operadores, visualizar interferencias cuánticas y distinguir entre estados clásicos y no clásicos.
\comentario{Aunque $W(x,p)$ se asemeja a una densidad de probabilidad, su capacidad de adoptar valores negativos revela la naturaleza cuántica subyacente.}

\subsection{Definición general}
Para un operador de densidad $\hat{\rho}$, la función de Wigner se define como:
\begin{tcolorbox}[bluebox]
\begin{equation}
  \begin{aligned}
    W(x,p) = \frac{1}{\pi\hbar} \int_{-\infty}^{\infty}
    \langle x+y | \hat{\rho} | x-y \rangle \, e^{-2ipy/\hbar} \, dy
  \end{aligned}
\end{equation}
\end{tcolorbox}
\comentario{La integral efectúa una transformada de Fourier sobre la diferencia de coordenadas, combinando la información de posición y momento.}

\subsection{Valores esperados en la representación de Wigner}
El valor esperado de un observable $\hat{A}$ puede calcularse mediante su símbolo de Wigner–Weyl $A_W(x,p)$:
\begin{tcolorbox}[bluebox]
\begin{equation}
  \begin{aligned}
    \langle \hat{A} \rangle = \iint W(x,p)\,A_W(x,p)\,dx\,dp
  \end{aligned}
\end{equation}
\end{tcolorbox}
\comentario{Esta expresión traduce la mecánica cuántica a un formalismo algebraico análogo al de la estadística clásica, pero con correcciones no conmutativas.}

\subsection{Ejemplo: Energía media del oscilador}
El símbolo de Wigner–Weyl del Hamiltoniano del oscilador armónico es:
\begin{tcolorbox}[bluebox]
\begin{equation}
  \begin{aligned}
    H_W(x,p) = \frac{p^2}{2m} + \frac{1}{2}m\omega^2x^2
  \end{aligned}
\end{equation}
\end{tcolorbox}
Para un estado coherente $|\alpha\rangle$, el valor esperado de la energía resulta:
\begin{tcolorbox}[bluebox]
\begin{equation}
  \begin{aligned}
    \langle \hat{H} \rangle_\alpha
    &= \iint W_\alpha(x,p)\,H_W(x,p)\,dx\,dp
    = \hbar\omega\left(|\alpha|^2 + \frac{1}{2}\right)
  \end{aligned}
\end{equation}
\end{tcolorbox}
\comentario{Este resultado coincide con la expresión obtenida directamente en el espacio de Hilbert, validando la equivalencia de ambos enfoques.}

\subsection{Criterios de no-clasicidad}
La presencia de regiones negativas en la función de Wigner indica que el estado cuántico no posee una descripción clásica.  
Este comportamiento se observa en los estados comprimidos intensamente o en superposiciones tipo “gato de Schrödinger”.
\begin{tcolorbox}[bluebox]
\begin{equation}
  \begin{aligned}
    W(x,p) < 0 \quad \Rightarrow \quad \text{Estado no clásico.}
  \end{aligned}
\end{equation}
\end{tcolorbox}
\comentario{Las regiones negativas surgen por interferencia cuántica, una propiedad imposible de reproducir con distribuciones probabilísticas reales.}

\subsection{Simetrías y transformaciones}
La función de Wigner obedece reglas sencillas bajo transformaciones lineales:
\begin{tcolorbox}[bluebox]
\begin{equation}
  \begin{aligned}
    W_\alpha(x,p) &\longrightarrow W_\alpha(x - x_0, p - p_0)
  \end{aligned}
\end{equation}
\end{tcolorbox}
\comentario{Un desplazamiento en el espacio de fase corresponde a la acción del operador de desplazamiento $\hat{D}(\alpha)$.}
Bajo una rotación por un ángulo $\theta$:
\begin{tcolorbox}[bluebox]
\begin{equation}
  \begin{aligned}
    W_\alpha(x,p) &\longrightarrow W_\alpha(x\cos\theta + p\sin\theta, -x\sin\theta + p\cos\theta)
  \end{aligned}
\end{equation}
\end{tcolorbox}
\comentario{Esta propiedad refleja la invariancia del formalismo frente a las rotaciones en el espacio de fase, análoga a la simetría del Hamiltoniano del oscilador.}

\subsection{Reconstrucción experimental}
La función de Wigner puede reconstruirse experimentalmente mediante \textit{tomografía cuántica}, en la que se miden las distribuciones de probabilidad de distintas cuadraturas $\hat{X}_\theta$.
\begin{tcolorbox}[bluebox]
\begin{equation}
  \begin{aligned}
    W(x,p) = \frac{1}{2\pi}\int_0^\pi P(X_\theta,\theta)\, e^{iR(\theta)} \, d\theta
  \end{aligned}
\end{equation}
\end{tcolorbox}
\comentario{En la práctica, $P(X_\theta,\theta)$ se obtiene mediante detección homodina, y el conjunto de mediciones permite reconstruir la función completa $W(x,p)$.}

\subsection{Visualización y aplicaciones}
La representación de Wigner se utiliza para analizar:
\begin{itemize}
  \item La coherencia y pureza de estados cuánticos.
  \item Los efectos de decoherencia y ruido térmico.
  \item La evolución temporal de paquetes de onda y campos ópticos.
\end{itemize}
\comentario{El análisis en el espacio de fase proporciona una visión geométrica del comportamiento cuántico, unificando descripciones clásicas y cuánticas.}
