\section{Espacio de Fase Cuántico}
El \textit{espacio de fase cuántico} constituye una representación intermedia entre la descripción de operadores y la descripción por funciones de onda.  
En él, un estado cuántico se expresa como una distribución sobre las variables continuas de posición y momento (o sus equivalentes en cuadraturas).
\comentario{El espacio de fase permite visualizar el estado cuántico de manera análoga a las distribuciones clásicas, pero respetando las restricciones impuestas por el principio de incertidumbre.}

\subsection{Motivación y concepto general}
En mecánica clásica, el estado de una partícula está completamente determinado por un punto $(x,p)$ en el espacio de fase.  
En mecánica cuántica, la imposibilidad de definir simultáneamente $x$ y $p$ con precisión obliga a reemplazar el punto por una \textit{función cuasi-probabilidad}.

\subsection{Función de Wigner}
La \textit{función de Wigner}, introducida en 1932, es la representación más común en el espacio de fase cuántico.  
Se define para un estado descrito por la función de onda $\psi(x)$ como:
\begin{tcolorbox}[bluebox]
\begin{equation}
  \begin{aligned}
    W(x,p) = \frac{1}{\pi\hbar} \int_{-\infty}^{\infty} \psi^*\left(x + y\right) \psi\left(x - y\right) e^{2ipy/\hbar} \, dy
  \end{aligned}
\end{equation}
\end{tcolorbox}
\comentario{Esta función combina información de posición y momento, pero puede adoptar valores negativos debido a la interferencia cuántica.}
\\La función de Wigner está normalizada de modo que:
\begin{tcolorbox}[bluebox]
\begin{equation}
  \begin{aligned}
    \iint W(x,p) \, dx \, dp = 1
  \end{aligned}
\end{equation}
\end{tcolorbox}
y sus proyecciones reproducen las densidades de probabilidad ordinarias:
\begin{tcolorbox}[bluebox]
\begin{equation}
  \begin{aligned}
    \int W(x,p)\, dp = |\psi(x)|^2, \qquad \int W(x,p)\, dx = |\phi(p)|^2
  \end{aligned}
\end{equation}
\end{tcolorbox}
\comentario{De este modo, la función de Wigner actúa como un puente entre las representaciones de posición y momento.}

\subsection{Propiedades generales}
Entre las propiedades más relevantes de $W(x,p)$ se encuentran:
\begin{enumerate}
  \item Es real, aunque no positiva definida.
  \item Está normalizada a la unidad.
  \item Se transforma covariantemente bajo traslaciones y rotaciones del espacio de fase.
  \item Permite calcular el valor esperado de un observable $\hat{A}$ mediante:
  \begin{tcolorbox}[bluebox]
  \begin{equation}
    \begin{aligned}
      \langle \hat{A} \rangle = \iint W(x,p)\, A_W(x,p)\, dx\,dp
    \end{aligned}
  \end{equation}
  \end{tcolorbox}
  donde $A_W(x,p)$ es el símbolo de Wigner–Weyl del operador $\hat{A}$.
\end{enumerate}

\subsection{Ejemplo: estado coherente en el espacio de fase}
Para un estado coherente $|\alpha\rangle$, la función de Wigner adopta una forma gaussiana centrada en $(x_0,p_0)$:
\begin{tcolorbox}[bluebox]
\begin{equation}
  \begin{aligned}
    W_\alpha(x,p) = \frac{1}{\pi\hbar}\exp\!\left[-\frac{(x-x_0)^2}{\sigma_x^2} - \frac{(p-p_0)^2}{\sigma_p^2}\right]
  \end{aligned}
\end{equation}
\end{tcolorbox}
\comentario{Esta expresión refleja que el estado coherente es el más "clásico" de todos los estados cuánticos, pues su distribución se mantiene mínima y centrada a lo largo del tiempo.}

\subsection{Simetría de paridad y negatividad cuántica}
Las regiones donde $W(x,p)<0$ no poseen interpretación probabilística clásica; representan interferencias puramente cuánticas.  
Esta característica hace de $W(x,p)$ una herramienta esencial para distinguir entre estados cuánticos clásicos y no clásicos.
\comentario{Más adelante se mostrará cómo los estados comprimidos y los estados de Fock de orden alto exhiben regiones de negatividad en $W(x,p)$.}

\subsection{Transformación de Wigner–Weyl}
El formalismo de Wigner puede extenderse a cualquier operador mediante la transformación de Wigner–Weyl:
\begin{tcolorbox}[bluebox]
\begin{equation}
  \begin{aligned}
    A_W(x,p) = \int e^{2ipy/\hbar}\left\langle x - y \middle| \hat{A} \middle| x + y \right\rangle dy
  \end{aligned}
\end{equation}
\end{tcolorbox}
\comentario{Esta transformación asocia a cada operador $\hat{A}$ una función en el espacio de fase, lo que permite expresar la mecánica cuántica en un marco algebraico completamente clásico.}
