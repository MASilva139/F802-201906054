\section{Evolución Temporal del Oscilador Armónico Cuántico}
El estudio de la evolución temporal permite comprender cómo los estados cuánticos del oscilador armónico se comportan bajo la acción del operador de evolución unitario.  
En el caso de estados coherentes o comprimidos, esta evolución mantiene la forma de la función de onda, modificando únicamente su fase y posición en el espacio de fase.
\comentario{El oscilador armónico es uno de los pocos sistemas cuánticos cuya evolución temporal puede obtenerse de forma exacta.}

\subsection{Ecuación de Schrödinger dependiente del tiempo}
La dinámica de cualquier estado cuántico $|\psi(t)\rangle$ está gobernada por la ecuación de Schrödinger:
\begin{tcolorbox}[bluebox]
\begin{equation}
  \begin{aligned}
    i\hbar \frac{d}{dt}|\psi(t)\rangle = \hat{H}|\psi(t)\rangle
  \end{aligned}
\end{equation}
\end{tcolorbox}
Para el oscilador armónico,
\begin{tcolorbox}[bluebox]
\begin{equation}
  \begin{aligned}
    \hat{H} = \hbar\omega\left(\hat{a}^\dagger \hat{a} + \frac{1}{2}\right)
  \end{aligned}
\end{equation}
\end{tcolorbox}
La solución formal de esta ecuación se expresa mediante el operador de evolución unitario:
\begin{tcolorbox}[bluebox]
\begin{equation}
  \begin{aligned}
    \hat{U}(t) = e^{-i\hat{H}t/\hbar}
  \end{aligned}
\end{equation}
\end{tcolorbox}
\comentario{El operador $\hat{U}(t)$ preserva la norma del estado, garantizando la conservación de la probabilidad total.}

\subsection{Evolución de los operadores de creación y aniquilación}
En el marco de Heisenberg, los operadores evolucionan en el tiempo según:
\begin{tcolorbox}[bluebox]
\begin{equation}
  \begin{aligned}
    \hat{a}(t) &= e^{i\hat{H}t/\hbar}\, \hat{a}\, e^{-i\hat{H}t/\hbar}, \\
    \hat{a}^\dagger(t) &= e^{i\hat{H}t/\hbar}\, \hat{a}^\dagger\, e^{-i\hat{H}t/\hbar}
  \end{aligned}
\end{equation}
\end{tcolorbox}
Evaluando las conmutaciones se obtiene:
\begin{tcolorbox}[bluebox]
\begin{equation}
  \begin{aligned}
    \hat{a}(t) &= \hat{a}(0)e^{-i\omega t}, \\
    \hat{a}^\dagger(t) &= \hat{a}^\dagger(0)e^{i\omega t}
  \end{aligned}
\end{equation}
\end{tcolorbox}
\comentario{El operador de aniquilación rota en el espacio de fase con frecuencia $\omega$, describiendo un movimiento circular.}

\subsection{Evolución del estado coherente}
Aplicando el operador de evolución sobre un estado coherente inicial $|\alpha(0)\rangle$, se tiene:
\begin{tcolorbox}[bluebox]
\begin{equation}
  \begin{aligned}
    |\alpha(t)\rangle &= \hat{U}(t)|\alpha(0)\rangle = e^{-i\omega t/2}|\alpha(0)e^{-i\omega t}\rangle
  \end{aligned}
\end{equation}
\end{tcolorbox}
\comentario{El estado coherente conserva su forma, pero rota en el espacio de fase siguiendo la trayectoria clásica del oscilador.}

\subsection{Evolución del estado comprimido}
Para un estado comprimido coherente $|\alpha,\zeta\rangle$, la evolución temporal introduce una rotación tanto del desplazamiento como del parámetro de compresión:
\begin{tcolorbox}[bluebox]
\begin{equation}
  \begin{aligned}
    |\alpha,\zeta;t\rangle = \hat{U}(t)|\alpha,\zeta\rangle
    = e^{-i\omega t/2}\, |\alpha e^{-i\omega t}, \zeta e^{-2i\omega t}\rangle
  \end{aligned}
\end{equation}
\end{tcolorbox}
\comentario{El parámetro $\zeta$ rota con el doble de la frecuencia angular, lo que implica una oscilación periódica de la elipse de incertidumbre en el espacio de fase.}

\subsection{Propagador del oscilador armónico}
Otra forma equivalente de estudiar la evolución es mediante el \textit{propagador} $K(x,t;x',0)$, que relaciona la función de onda inicial con la final:
\begin{tcolorbox}[bluebox]
\begin{equation}
  \begin{aligned}
    \psi(x,t) = \int_{-\infty}^{\infty} K(x,t;x',0)\,\psi(x',0)\,dx'
  \end{aligned}
\end{equation}
\end{tcolorbox}
El propagador exacto del oscilador armónico se obtiene evaluando la integral de camino y resulta:
\begin{tcolorbox}[bluebox]
\begin{equation}
  \begin{aligned}
    K(x,t;x',0) &= \sqrt{\frac{m\omega}{2\pi i\hbar\sin(\omega t)}}
    \exp\!\left[\frac{i m\omega}{2\hbar\sin(\omega t)}\left((x^2+x'^2)\cos(\omega t) - 2xx'\right)\right]
  \end{aligned}
\end{equation}
\end{tcolorbox}
\comentario{Este propagador describe cómo la probabilidad se propaga en el tiempo dentro del potencial armónico. Es una herramienta central en formulaciones de la mecánica cuántica basadas en integrales de camino.}

\subsection{Periodicidad y estabilidad del movimiento}
El oscilador armónico cuántico exhibe periodicidad exacta en su evolución: después de un periodo $T = 2\pi/\omega$, todos los estados recuperan su forma inicial salvo una fase global.
\begin{tcolorbox}[bluebox]
\begin{equation}
  \begin{aligned}
    |\psi(T)\rangle = e^{-i\pi}|\psi(0)\rangle
  \end{aligned}
\end{equation}
\end{tcolorbox}
\comentario{La fase adquirida, llamada \textit{fase de Berry dinámica}, carece de efectos observables, pero tiene implicaciones topológicas en sistemas más complejos.}
